\documentclass[11pt,a4paper]{article}
\usepackage[utf8]{inputenc}
\usepackage[russian]{babel}
\usepackage{graphicx}
\usepackage{amsmath,amssymb}
\usepackage{geometry}
\usepackage{setspace}
\usepackage{fancyhdr}

\geometry{a4paper, left=35mm, right=35mm, top=30mm, bottom=30mm}
\renewcommand{\baselinestretch}{0.9}
\DeclareMathOperator*{\smallSum}{\textstyle\sum}
\setcounter{page}{347}


\cfoot{\rule{60}{0.1pt} \\ \thepage}
\pagestyle{fancy}



\begin{document}
\begin{Large}
\noindent\sffamily{\textbf{13.3. Формулы Тейлора \\ для основных элементарных функций \\}}

\noindent 1. f(x) = sin\:x. Функция $sin\:x$ обладает производными всех порядков. Найдём для неё формулу Тейлора при \(x_0 = 0\), т.е. формулу Маклорена (13.8).
Было доказано (см. п. 10.1), что
$(sin\:x)^{(m)} = sin\:\Big(x + m\frac{\pi}{2}\Big)$, поэтому
\renewcommand{\baselinestretch}{0.5}
$$f^{(m)}(0) = sin\:\frac{m\pi}{2}= \scalebox{0.9}{$\begin{cases}
		0 & \text{для $m = 2k$}, \\
		(-1)^k & \text{для $m = 2k+1$},
	\end{cases}$}
	\text{$k = 0, 1, 2,..., $ (13.16)} $$

\noindent и, согласно формуле $(13.5)$, 
\begin{align*}
 sin \: x = x - \frac{x^3}{3!} + \frac{x^5}{5!} - \frac{x^7}{7!} + ... + (-1)^n \frac{x^{2n+1}}{(2n+1)!}+o(x^{2n+2}),
\end{align*}
\noindent при $x \rightarrow 0, n = 0, 1, 2, ...,$ или, короче,
\begin{align*} 
sin \: x = \smallSum_{k=0}^{n} (-1)^k \frac{x^{2n+1}}{(2n+1)!}+o(x^{2n+2}) \text{ при } x \rightarrow 0
\end{align*}

\noindent Мы записали здесь остаточный член в виде $o(x^{2n+2}) $, а не в виде $o(x^{2n+1})$, так как следующий за последним выписанным слагаемым
член многочлена Тейлора, в силу (13.16), равен нулю.

2. $f(x) = cos\:x$. Как известно (см. п. 10.1),
\renewcommand{\baselinestretch}{0.5}
\begin{align*} 
f^{(m)}(x) = cos\:(x + \frac{m\pi}{2})
\end{align*} 
поэтому
$$f^{(m)}(0) = cos\:\frac{m\pi}{2}= \scalebox{0.9}{$\begin{cases}
		0 & \text{для $m = 2k+1$}, \\
		(-1)^k & \text{для $m = 2k$},
	\end{cases}$}
	\text{$k = 0, 1, 2,..., $} $$
и
\begin{align*}
 cos \: x = 1 - \frac{x^2}{2!} + \frac{x^4}{4!} - \frac{x^6}{6!} + ... + (-1)^n \frac{x^{2n}}{(2n)!}+o(x^{2n+1}),
\end{align*}
при $x \rightarrow 0,$ или, короче,
\begin{align*}
cos \: x = \smallSum_{k=0}^{n} (-1)^k \frac{x^{2k}}{(2k)!}+o(x^{2n+1}) 
\end{align*}
при $x \rightarrow 0$, n = 0, 1, 2, ... .

3. $f(x) = e^x$. Так как $(e^x)^{(n)} = e^x$, то $f^{(n)}(0) = 1, n = 0, 1, ... ,$ следовательно,
\begin{equation} \tag{13.17}
e^x  = 1 + x + \frac{x^2}{2!} + \frac{x^3}{3!} + ... + \frac{x^{n}}{n!}+o(x^n),
\end{equation}
\end{Large}
\end{document}